Methoden van Jacobi en Gauss-Seidel convergeren enkel indien de matrix A van het stelsel \textit{diagonaal dominant is}.\\
Dus: een element op de diagonaal moet in absolute waarde groter zijn dan de som van de absolute waarden van alle andere elementen die zich op \textit{dezelfde rij} bevinden. Dit is voldoende maar niet altijd \textit{nodig} voor convergentie.\\\\
Dus:\\
$|\alpha+1| > 1$ en\\
$1 > |\alpha|$\\\\
Dit geldt voor:
$\alpha \in ]0,1[$\\
En voor: 
$\alpha \in ]-2,-\infty[$ (Deze was vergeten in de wiki-oplossingen)